% Define document class
\documentclass[twocolumn]{aastex631}

% Filler text
\usepackage{blindtext}

% Begin!
\begin{document}

% Title
\title{An open source scientific article}

% Author list
\author{First Author}

% Abstract with filler text
\begin{abstract}
    \blindtext
\end{abstract}

% Main body
\section{Introduction}

Figure~\ref{fig:my_figure} is an example of a figure that can be generated one of two different ways.
This is useful in the case that the figure is the result of, e.g., a very expensive simulation or MCMC run, which you might not want to ever run on GitHub Actions!
Therefore, two rules exist for generating the figure, and which one gets run depends on whether the code is running on GitHub Actions or locally.
If the code is run locally, the figure is generated from scratch by running the expensive simulation.
The result of the expensive simulation is then uploaded to Zenodo.
Conversely, if the code is run on GitHub Actions, the code downloads the simulation results (skipping the expensive step) and then generates the figure.
Check out the \texttt{Snakefile} and the \texttt{showyourwork.yml} config file for the full implementation of these rules.

% A sample figure generated from an external dataset
\begin{figure}[ht!]
    \begin{centering}
        \includegraphics[width=0.75\linewidth]{figures/my_figure.pdf}
        \caption{
            A sample figure generated from the output of a very expensive simulation.
        }
        % This label tells showyourwork that the script `figures/my_figure.py'
        % generates the PDF included above
        \label{fig:my_figure}
    \end{centering}
\end{figure}

\end{document}
