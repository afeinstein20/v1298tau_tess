\documentclass[twocolumn]{aastex631}

\usepackage{xspace}

\usepackage{xcolor, fontawesome}

\newcommand{\tess}{\textit{TESS}}
\newcommand{\sname}{V1298~Tau\xspace}
\newcommand{\allplanets}{V1298~Tau~bcd\xspace}
\newcommand{\planetb}{V1298~Tau~b\xspace}
\newcommand{\planetc}{V1298~Tau~c\xspace}
\newcommand{\planetd}{V1298~Tau~d\xspace}
\newcommand{\planete}{V1298~Tau~e\xspace}
\newcommand{\rearth}{$R_\oplus$\xspace}
\newcommand{\exoplanet}{\texttt{exoplanet}\xspace}



\submitjournal{AJ}


\shorttitle{V1298 Tau with \tess}
\shortauthors{Feinstein et al.}


\begin{document}

\title{V1298~Tau with TESS: Updated Ephemerides, Radii, and Period Constraints for a Previous Single Transit Event of V1298~Tau~e}

\author[0000-0002-9464-8101]{Adina~D.~Feinstein}
\altaffiliation{NSF Graduate Research Fellow}
\affiliation{Department of Astronomy and Astrophysics, University of Chicago, Chicago, IL 60637, USA}

\author[0000-0001-6534-6246]{Trevor J.\ David}
\affiliation{Center for Computational Astrophysics, Flatiron Institute, New York, NY 10010, USA}
\affiliation{Department of Astrophysics, American Museum of Natural History, New York, NY 10024, USA}

\author[0000-0001-7516-8308]{Benjamin~T.~Montet}
\affiliation{School of Physics, University of New South Wales, Sydney, NSW 2052, Australia}
\affiliation{UNSW Data Science Hub, University of New South Wales, Sydney, NSW 2052, Australia}

\author[0000-0002-9328-5652]{Daniel Foreman-Mackey}
\affiliation{Center for Computational Astrophysics, Flatiron Institute, New York, NY 10010, USA}

\author[0000-0002-4881-3620]{John~H.~Livingston}
\affiliation{Department of Astronomy, University of Tokyo, 7-3-1 Hongo, Bunkyo-ku, Tokyo 113-0033, Japan}

%\author{Charles Beichman}
%\affiliation{Caltech/IPAC, 1200 E. California Blvd. Pasadena, CA 91125, USA}

\author[0000-0002-3199-2888]{Sarah Blunt}
\altaffiliation{NSF Graduate Research Fellow}
\affiliation{Department of Astronomy, California Institute of Technology, Pasadena, CA, USA}



\correspondingauthor{Adina~D.~Feinstein;\\ \twitter{afeinstein20}; \github{afeinstein20};} \email{afeinstein@uchicago.edu} 

%%%%%%%%%%%%%%%%%%%%
% abstract must be < 250 words
% currently at 180 words
%%%%%%%%%%%%%%%%%%%%
\begin{abstract}
\sname is a young (20--30~Myr) solar analogue hosting four transiting exoplanets with sizes between $0.5 - 0.9 R_J$. Given the system's youth,  it provides a unique opportunity to understand the evolution of planetary radii in the same stellar environment. \sname was originally observed 6 years ago during K2 Campaign 4. The extended mission of NASA's Transiting Exoplanet Survey Satellite (\tess) includes observing the ecliptic plane. Here, we present new photometric observations of \sname from the 10-minute TESS Full-Frame Images. We use the TESS data to update the transit-timing for \allplanets as well as compare newly observed radii to those derived from the K2 light curve, finding shallower transits in the redder \tess\ bandpass at the $1-2\sigma$ level. We hypothesize the shallower transits observed by \tess\ may be due to high-altitude hazes in the extended atmospheres of these young planets. We additionally catch a transit of \planete, which was only observed once during the \textit{K2} observations, and present a novel method for deriving the marginalized posterior probability of a planet from two transits observed years apart. We find the highest probability period for \planete to be in a near 2:1 mean motion resonance with \planetb which, if confirmed, would make this a nearly 4 planet resonant chain. \sname is the target of several ongoing and future observations, including with the James Webb Space Telescope. These updated ephemerides will be crucial for accurately recovering transit events and understanding any future Doppler tomographic or transmission spectroscopy signals.\end{abstract}

%%%%%%%%%%%%%%%%%%%%

\keywords{Exoplanets (498) --- Pre-main sequence (1289) --- Starspots (1572) --- Stellar activity (1580)}

%%%%%%%%%%%%%%%%%%%%

\section{Introduction} \label{sec:intro}
Planetary sizes are expected to evolve over time, due to a variety of endogenous and exogenous physical processes including gravitational contraction, atmospheric heating and mass-loss, and core-envelope interactions \citep[e.g.][]{OwenWu2013, Lopez2013, Jin2014, ChenRogers2016, Ginzburg2018}. Dramatic size evolution is expected at early stages, when planets are still contracting and radiating away the energy from their formation, and when host stars are heating planetary atmospheres with high levels of X-ray and ultraviolet radiation. Since the size evolution of any individual planet is believed to be slow relative to typical observational baselines, the best way to make inferences about the size evolution of exoplanets is by measuring the sizes of large numbers of planets across a range of ages. 

NASA's Transiting Exoplanet Survey Satellite \citep[TESS,][]{Ricker2015} has made significant inroads toward this objective. \tess's observations of $\sim 90 \%$ of the sky have allowed for exoplanet transit searches around stars ranging from the pre-main sequence to the giant branch. It is through targeted surveys of young stars such as the THYME \citep[e.g.][]{Newton2019}, PATHOS \citep[e.g.][]{Nardiello2020}, and CDIPS \citep[e.g.][]{Bouma2020} surveys, along with case studies of individual systems \citep[e.g.][]{benatti19, Plavchan2020, Hedges2021, Zhou2021} that the timeline for planetary radius evolution can be pieced together. 

The \sname planetary system is one particularly valuable benchmark for understanding the size evolution of exoplanets. \sname is a pre-main sequence, approximately solar-mass star that was observed in 2015 by NASA's \textit{K2} mission \citep{Howell2014}. Analysis of the \textit{K2} data revealed the presence of four transiting planets, all with sizes between the sizes of Neptune and Jupiter \citep{David2019a, David2019b}. There are no other known examples of exoplanetary systems with so many planets larger than Neptune interior to 0.5~au, despite the high completeness of the Kepler survey to large ($>5$~\rearth), close-in planets. This observation raises the possibility of a causal connection between the extreme youth of \sname and the uncommonly large sizes of its planets.  

The youth of \sname, also known as K2-309 and [WKS96] 4, was initially established on the basis of its strong X-ray emission \citep{Wichmann1996} and high photospheric lithium abundance \citep{Wichmann2000}. A blind search for co-moving stars using Gaia DR1 astrometry data found \sname was co-moving with 8 other stars \citep[Group 29 in][]{Oh2017}. In a kinematic study of the Taurus star-forming region \citet{Luhman2018} used Gaia DR2 data to extended the membership of this group and derived an age of $\sim$~40~Myr. However, more recent analyses based on Gaia EDR3 astrometry suggests \sname may belong to either the D2 or D3 subgroups of Taurus, both of which have estimated ages $\lesssim$10~Myr \citep{gaidos21, Krolikowski2021}. Other studies focused specifically on the \sname system have estimated its age to be 23$\pm$4~Myr from comparison with empirical and theoretical isochrones \citep{David2019b}, or 28$\pm$4~Myr from isochrone fitting to the \citet{Luhman2018} Group 29 membership list given Gaia EDR3 data \citep{johnson21}. While the precise age of \sname remains uncertain, most estimates fall in the 10--40~Myr range and we adopt $t \approx$~20--30~Myr.

Given the system's youth and potential to reveal information about the initial conditions of close-in planetary systems \citep[e.g.][]{Poppenhaeger2021, Owen2020}, \sname has been the target for extensive follow-up observations. These include efforts to constrain planet masses with radial velocities \citep{Beichman2019}, measure the spin-orbit alignments of planet c \citep{Feinstein21} and planet b (Johnson et al. submitted; Gaidos et al. submitted), measure or constrain atmospheric mass-loss rates for the innermost planets \citep{Schlawin21, Vissapragada21}, and an approved program to study the planetary atmospheres using the James Webb Space Telescope \citep{Desert2021}.


Here we report on newly acquired TESS observations of \sname which help to refine the orbital ephemerides of the transiting planets and enable comparison of the planet sizes inferred from two different telescopes with different bandpasses (\tess\ and Kepler). We describe the observations and light curve extraction in Section~\ref{sec:observations}. In Section~\ref{sec:analysis}, we present our transit and stellar variability modeling. We additionally present a novel method for computing the marginalized posterior probability of a transiting planet's period from two transits observed with a large gap between them. We discuss the differences in measured transit parameters between \textit{K2} and \tess\ data. We conclude in Section~\ref{sec:conclusions}.

%%%%%%%%%%%%%%%%%%%%%%%%%%%%%%%%%%%%%%%%%%%%%%%%%%%%%%%
\begin{figure}[!ht]
\begin{center}
\includegraphics[width=0.46\textwidth,trim={0.25cm 0 0 0}]{static/TESSaperture.pdf}
\caption{The \tess\ target pixel file (TPF) overlaid with a sky image of \sname taken with the Sloan Digital Sky Survey (SDSS) i-band. \sname is shown in the white circle; nearby sources with \tess\ magnitudes $< 14$ are marked with orange x's. There are two bight nearby sources to \sname. However, selecting the one pixel aperture (white square) encapsulates only \sname and therefore contamination from the nearby sources is negligible. \href{https://github.com/afeinstein20/v1298tau_tess/blob/main/src/figures/tpf.py}{\github}} \label{fig:tpf}
\end{center}
\end{figure}
%%%%%%%%%%%%%%%%%%%%%%%%%%%%%%%%%%%%%%%%%%%%%%%%%%%%%%%

\section{TESS Observations} \label{sec:observations}

During its Extended Mission Cycle 4, \tess\ is re-observing many of the previous \textit{K2} fields. \sname (TIC 15756231) was observed by \tess\ in Sectors 43 (UT 16 Sep 2021 to UT 12 Oct 2021) at 10-minute cadence within the Full-Frame Images (FFIs). We used the \texttt{tica} \citep{fausnaugh20} software to download calibrated FFIs each orbit for Sectors 43, as these FFIs are quickly available after the data is down-linked. 

We created light curves from the \texttt{tica}-processed FFIs by modeling the point-spread function (PSF) of \sname and the two nearby bright sources (see Figure~\ref{fig:tpf}), following the PSF modeling routine in \cite{feinstein19}. In summary, we calculated and maximized the likelihood value of four parameters per each Gaussian: the $x$ and $y$ width, amplitude, and a rotational term. The Gaussian fits are allowed to vary at each time step. While aperture photometry provided a decent light curve, we found that modeling each star (3 in total) with a 2D Gaussian created a less contaminated light curve for \sname. Our extracted light curve is shown in the top row of Figure~\ref{fig:transits}.

%%%%%%%%%%%%%%%%%%%%%%%%%%%%%%%%%%%%%%%%%%%%%%%%%%%%%%%
\begin{figure*}[hbtp]
\begin{center}
\includegraphics[width=\textwidth,trim={0.25cm 0 0 0}]{figures/transits.pdf}
\caption{\sname extracted light curve from the \texttt{tica}-processed full-frame images for all of Sector 43 and the first orbit of Sector 44, with transits of \allplanets highlighted by color. Each subplot contains the raw, normalized \tess\ flux (top) and the GP model removed flux (de-trended flux; bottom). Top row: extracted light curve with over plotted with our best-fit GP model for stellar variability (black). Bottom three rows: zoomed-in regions around the transits present in the \tess\ data. The GP best-fit model is over-plotted on the raw, normalized flux. The best-fit transit models are over-plotted on the de-trended flux. For overlapping transits, the sum of the transits is blotted in black.} \label{fig:transits}
\end{center}
\end{figure*}
%%%%%%%%%%%%%%%%%%%%%%%%%%%%%%%%%%%%%%%%%%%%%%%%%%%%%%%

\section{Analysis} \label{sec:analysis}

% planet fits
We simultaneously modeled the transits of \allplanets and the stellar variability using the open-source packages \exoplanet \citep{exoplanet2019, exoplanet2021} and \texttt{PyMC3} \citep{Salvatier16}. Transit timings were originally identified using updated ephemerides from \textit{Spitzer} (Livingston et al. in prep). All other transit parameters (presented in Table~\ref{tab:table}) were initialized using values from \cite{David2019a}. We assumed quadratic limb darkening law, following the reparameterization described by \cite{kipping13}; this method allows for an efficient and uniform sampling of limb-darkening fits.

% light curve fits

Since the \texttt{tica} FFIs do not provide an error estimate, we fit for flux errors within our Gaussian Process (GP) model. We define the flux error as

\begin{equation}
    \sigma_y = e^{ln(\sigma_l)} + y^2 e^{2 ln(\sigma_j)}
\end{equation}

where $y$ is the normalized flux array about zero, and $\sigma_l$  and $\sigma_j$ are used to define the light curve noise and in-transit jitter. $\sigma_l$  and $\sigma_j$ are also used as the first and second terms in our rotation model, which we defined as a stochastically-driven, damped harmonic oscillator, defined by the \texttt{SHOTerm} in \texttt{celerite2} \citep{dfm17}.

We took a two-step approach to modeling any background contamination. First, we defined a quadratic trend with respect to time for varying the background flux, where each term was drawn from a normal distribution (listed as `trend term' in Table~\ref{tab:table}). Then, we generated a Vandermonde matrix ($A$) of time. This is a way of introducing a polynomial least-squares regression with respect to time. The final background flux was calculated by taking $bkg = A \cdot trend$.  
We performed an MCMC sampling fit to each parameter. We ran 3 chains with 500 tuning steps and 2000 draws. We discarded the tuning samples in our final modeling best-fits, presented in Table~\ref{tab:table}. We verified our chains converged via visual inspection and following the diagnostic provided by \cite{Geweke92}.

Our final GP model for stellar variability and planet transits are shown in Figure~\ref{fig:transits}. There is a $\sim 1\%$ flare at \tess\ BJD 4659.18 that we do not fit. Zoom-ins on the individual transits that occur are presented in the bottom two rows of Figure~\ref{fig:transits}. Phase-folded transits for \planetc and \planetd are presented in Figure~\ref{fig:compare}.


\subsection{Constraining \planete's Period}

There was a single transit of \planete in the original \textit{K2} data, which occurred roughly in the middle of the campaign. Since no other transits were detected, this provides a lower period limit of 36~days. Using the original transit timing from \textit{K2} and this new transit timing from \tess, we developed a new method for constraining the period of \planete. For this analysis, we used the \texttt{EVEREST 2.0} \citep{luger18} version of the \textit{K2} light curve. 

For this analysis, we created a GP model that constrains the period of \planete by sampling over discrete periods, ranging from $38 - 56$~days. We fit for the additional transit parameters, $\theta$, assuming a constant transit depth between the \textit{K2} and \tess\ observations, $X$. To speed up the computation of the GP fits, we localized a 1-day window around each transit of \planete.  

We assumed there is no correlation between the other transit parameters and the period we are fitting for. For each step in our MCMC fit, we compute all possible periods

\begin{equation}\label{eq:period}
    P = \frac{1}{q} \left(T_{mid,TESS} - T_{mid, K2}\right)
\end{equation}

where $T_{mid}$ are the transit midpoints from \textit{K2} and \tess\ and $q$ is an integer. We assume a uniform prior, i.e. we have no prior preference for a specific harmonic, $q$. At each step of the sampling process, we compute a new light curve with different orbital periods, given by Equation~\ref{eq:period}. The log likelihood of each individual generated light curve is saved

\begin{equation}
    \textrm{log} \mathcal{L}_q = \left[ \textrm{log}\, p \left( X | \theta^k, q^k = n \right) \right]_n
\end{equation}

along with the sum of all log likelihoods for each period 

\begin{equation}
    \textrm{log} \mathcal{L} = \textrm{log}\, \Sigma_q\, p(q)\, p(X|\theta^k, q) 
\end{equation}

The summation of all log likelihoods is used to compute the posterior probability for each sampled value of $q$.

We chose to test periods from $38-56$~days. We ran 5 chains with 500 tuning steps and 5000 draws. Similarly to the transit fits, we discarded the tuning steps before our analysis. We show our results in Figure~\ref{fig:period_e}, where we plot the median period for each tested harmonic against the posterior probability of each harmonic.

%%%%%%%%%%%%%%%%%%%%%%%%%%%%%%%%%%%%%%%%%%%%%%%%%%%%%%%
\begin{figure}[!ht]
\begin{center}
\includegraphics[width=0.45\textwidth,trim={0.25cm 0 0cm 0}]{figures/periode.pdf}
\caption{Our calculated log likelihood to constrain the period of \planete using transit timings from \textit{K2} and \tess. We tested discrete periods from $38-56$\,days and find the most likely period to be 50.748~days. The 2:1 resonance (48.28~days) with \planetb is plotted as the dashed vertical line.} \label{fig:period_e}
\end{center}
\end{figure}
%%%%%%%%%%%%%%%%%%%%%%%%%%%%%%%%%%%%%%%%%%%%%%%%%%%%%%%

Using this method, we find the most likely period of \planete to be 50.748~days. We computed a $\chi^2$-fit to a Gaussian distribution and find the period to be $50.29 \pm 6.62$~days. Even at the lower limit, it is unlikely a transit of \planete will occur in the next sector (Sector 44) of \tess\ data for \sname. However, this derived period estimate suggests that \planete is in a near 2:1 mean motion resonance with \planetb.

Independent ground-based monitoring of the system for several days beyond the end of Sector 44 (UT 06 Nov 2021) may be able to observe another transit of the outermost known planet in this system. We provide several transit dates and times that are observable from Mauna Kea, given the top 5 most-likely periods of \planete, in Table~\ref{tab:planete}

\section{Differences in Measured Radii} \label{sec:radii}

We compare the differences in transit $R_p/R_\star$ between the \textit{K2} and \tess\ data in Figure~\ref{fig:compare}. We plot the best-fit transit model from \textit{K2} in black, which was generated using parameters from \cite{David2019a} and the \texttt{batman} transit-modeling software \citep{Kreidberg15}. The residuals of the \tess\ light curve with our model and the \textit{K2} model are plotted as well. For each planet, the transit $R_p/R_\star$ is shallower in the \tess\ data (Figure~\ref{fig:compare}, bottom panel).

Because \sname is young, it is expected that \allplanets have extended atmospheres due to high levels of stellar irradiation \citep{OwenWu2017} and are slowly evolving as atmospheric mass is lost. Only a tentative mass-loss rate of \planetd has been constrained from observations of the metastable He\textsc{I} line, along with non-detections for \planetb and \planetc \citep{Vissapragada21}. As such, we favor that the shallower transit depth is due to the difference in bandpass wavelength coverage between \textit{K2} and \tess, rather than atmospheric evolution. The \textit{K2} bandpass ranges from $\sim 400-900$~nm \citep{Howell2014}, while the \tess\ bandpass ranges from $\sim 600-1000$~nm \citep{Ricker2015}. A shallower transit depth at redder wavelengths is supported by the dusty outflow model presented in \citep{wang19}.

%%%%%%%%%%%%%%%%%%%%%%%%%%%%%%%%%%%%%%%%%%%%%%%%%%%%%%%
\begin{figure}[!ht]
\begin{center}
\includegraphics[width=0.465\textwidth,trim={0.25cm 0 4.5cm 0}]{figures/folded_compare.pdf}
\caption{Phase-folded \tess\ data (gray) with the new best-fit model (color) compared to the best-fit parameters from the original \textit{K2} data (black). The residuals between the data and each fit are shown underneath. A comparison of measured $R_p/R_\star$ are presented at the bottom, with a 1-to-1 line plotted in black for reference. The transit depth from the \textit{K2} data is deeper than that from \tess, particularly for \planetb and \planete.} \label{fig:compare}
\end{center}
\end{figure}
%%%%%%%%%%%%%%%%%%%%%%%%%%%%%%%%%%%%%%%%%%%%%%%%%%%%%%%

The ability of a planet to retain hazes/transition hazes is positively correlated with its equilibrium temperature, $T_{eq}$, internal temperature $T_{int}$, and mass, $M_{core}$. \allplanets have calculated $T_{eq}<1000$\,K, assuming an albedo = 0 \citep{David2019a}. Young planets are believed to have high $T_{int}$ due to ongoing gravitational contraction \citep{gu04}. The combination of these two parameters make these planets more likely to have extended atmospheres, while outflow winds lead to the formation of transition hazes \citep{gao20}. 

If \allplanets host extended haze-dominated atmospheres, they should have smaller transit radii at mid-IR wavelengths \citep{gao20}. However, without mass estimates for \allplanets, it is difficult to determine if this is the cause of the different transit depths we measured between the \textit{K2} and \tess\ data. While radial velocity mass measurements are challenging for young planets due to the underlying stellar activity, efforts should be taken to try and constrain the planet masses. 

Another, perhaps more promising approach, would be to derive the planet masses using transit-timing variations \citep[TTVs;][]{agol18}. A full TTV analysis using \textit{K2}, \textit{Spitzer}, and \tess\ transit photometry is presented in Livingston et al. (in prep). 

\section{Conclusions} \label{sec:conclusions}

We present new ephemerides for all four known planets in the \sname system. We detected a second transit of \planete; this new transit time in combination from the transit observed with \textit{K2} allowed us to place tighter constraints on the period of the outermost planet.

We find the transit depths of \allplanets as observed by \tess\ are shallower than previous \textit{K2} by $1-2\sigma$. We postulate this could be due to ongoing dusty outflows, that make the transit depth appear shallower. However, measuring the transit depths at longer wavelengths could corroborate this.


The Python code used to access the data, process and model the light curve, and reproduce the tables and figures are made publicly available.\footnote{\url{https://github.com/afeinstein20/v1298tau_tess}}

\begin{acknowledgments}
We thank Rodrigo Luger for developing \texttt{showyourwork!} \citep{luger21} and helping us debug this letter and Chas Beichman for helpful comments on our \tess\ DDT proposal 36. It is a pleasure to thank the Astronomical Data Group at the Flatiron Institute for helpful discussions. ADF acknowledges support from the National Science Foundation Graduate Research Fellowship Program under Grant No. (DGE-1746045).

This research has made use of NASA's Astrophysics Data System Bibliographic Services. This research made use of Lightkurve, a Python package for \textit{Kepler} and \tess\ data analysis \citep{lightkurve}.

\end{acknowledgments}


\begin{deluxetable*}{l r r r r}[!ht]
\tabletypesize{\footnotesize}
\tablecaption{\sname light curve fitting results. \label{tab:table}}
\tablehead{\\
\hline\
\textit{Star} & \textit{Value} & \textit{Prior} &  & \\
\hline
$R_\star [R_\odot]$ & $1.354_{-0.037}^{+0.039}$ & $\mathcal{G}(1.305, 0.07)$ & & \\
$M_\star [M_\odot]$ & $1.088_{-0.047}^{+0.049}$ & $\mathcal{G}(1.10, 0.05)$& & \\
$u_1$ & $0.39_{-0.14}^{+0.13}$ & $\mathcal{U}[0, 1]$ in $q_1$ & & \\
$u_2$ & $0.02_{-0.17}^{+0.16}$ & $\mathcal{U}[0, 1]$ in $q_2$ & & \\
$P_{rot}$ [days] & $2.97 \pm 0.03$ & $\mathcal{G}($ln$ 2.87, 2)$ & & \\
ln($Q_0$) & $0.65 \pm 0.19$ & $\mathcal{H}(\sigma=2)$ & & \\
$\Delta Q_0$ & $4.51 \pm 0.45$ & $\mathcal{G}(0, 2)$ & & \\
f [ppt] & $0.74_{-0.13}^{+0.12}$ & $\mathcal{U}[0.1, 1]$ & & \\
\hline\
\textit{Light Curve} & \textit{Value} & \textit{Prior} &  & \\
\hline
$\mu$ & $-1.6 \pm 1.0$ & $\mathcal{G}(0, 10)$ & &\\
ln($\sigma_l$) & $-1.94 \pm 1.73$ & $\mathcal{G}($ln$(0.1\sigma_\textrm{flux}), 10)$ & & \\
ln($\sigma_j$) & $-1.31 \pm 1.14$ & $\mathcal{G}($ln$(0.1\sigma_\textrm{flux}), 10)$ & & \\
\hline\
\textit{Planets} & \textit{c} & \textit{d} & \textit{b} & \textit{e}\\
\hline
$T_0$ [BKJD - 2454833]  & $4648.1571_{-0.0034}^{+0.0039}$ & $4645.4159_{-0.0028}^{+0.0025}$ & $4648.0926_{-0.0017}^{+0.0015}$ & $4648.8002_{-0.0017}^{+0.0017}$ \\
P [days]               & $8.2484_{-0.002}^{+0.002}$ & $12.4006_{-0.002}^{+0.002}$ & $24.1372_{-0.0021}^{+0.0022}$ & $50.74 \pm 5.88$ \\
$R_p/R_\star$          &  $0.0358_{-0.0005}^{+0.0004}$ & $0.041_{-0.0006}^{+0.0005}$ & $0.067_{-0.0007}^{+0.0005}$ & $0.0587_{-0.0006}^{+0.0005}$ \\
$b$                    &  $0.25 \pm 0.09$ & $0.31_{-0.12}^{+0.08}$ & $0.50_{-0.05}^{+0.04}$ & $0.53_{-0.06}^{+0.05}$ \\
T$_{14}$ [hours]        &  $4.67_{-0.44}^{+0.48}$ & $5.63_{-0.54}^{+0.59}$ & $6.39_{-0.6}^{+0.68}$ & $7.42_{-0.69}^{+0.79}$  \\
$R_p [R_\oplus]$       & $5.28 \pm 0.17$ & $6.06 \pm 0.21$ & $9.89 \pm 0.32$ & $8.67 \pm 0.28$\\
$R_p [R_J]$       & $0.47 \pm 0.15$ & $0.54 \pm 0.02$ & $0.88 \pm 0.03$ & $0.77 \pm 0.02$\\
\hline\
\textit{Priors} & \textit{c} & \textit{d} & \textit{b} & \textit{e}\\
\hline
$T_0$ [BKJD - 2454833] & $\mathcal{G}(4648.53,0.1)$ & $\mathcal{G}(4645.4,0.1)$ & $\mathcal{G}(4648.1,0.1)$ & $\mathcal{G}(4648.8,0.1)$ \\
log($P$ [days]) & $\mathcal{G}($ln$ 8.25, 1)$ & $\mathcal{G}($ln$ 12.40, 1)$& $\mathcal{G}($ln$ 24.14, 1)$ & $\mathcal{G}($ln$ 36.70, 1)$\\
log(depth [ppt]) & $\mathcal{G}($ln$ 1.45, 10^{-4})$ & $\mathcal{G}($ln$ 1.90, 10^{-4})$ & $\mathcal{G}($ln$ 4.90, 10^{-4})$ & $\mathcal{G}($ln$ 3.73, 10^{-4})$ \\
$b$ & $\mathcal{U}[0, 1]$ & $\mathcal{U}[0, 1]$ & $\mathcal{U}[0, 1]$ & $\mathcal{U}[0, 1]$ \\
T$_{14}$ [days] & $\mathcal{G}($ln$ 0.19, 1)$ & $\mathcal{G}($ln$ 0.23, 1)$ & $\mathcal{G}($ln$ 0.27, 1)$ &  $\mathcal{G}($ln$ 0.31, 1)$
}
\startdata
\enddata
\tablecomments{Priors are noted for parameters that were directly sampled. The distributions are as follows -- $\mathcal{G}$: Gaussian; $\mathcal{H}$: Half-normal; $\mathcal{U}$: Uniform. $\sigma_\textrm{flux}$ is the standard deviation of the light curve}
\end{deluxetable*}

\begin{deluxetable*}{l r r r r}[!ht]
\tabletypesize{\footnotesize}
\tablecaption{Predicted Ground-based Transit Midpoint Events for \planete \label{tab:planete}}
\tablehead{
\textit{P} [days] & Posterior Prob. & Mountain Time [UTC-7] & Pacific Time [UTC-8] & Hawaiian Time [UTC-10] \\
\hline
50.748 & $1.88 \times 10^{-4}$ & 13-11-2021 18:09 & --- & --- \\
    &  & --- & 23-02-2022 05:04 & 23-02-2022 03:04 \\
    &  & 15-04-2022 01:01 & 15-04-2022 00:01 & 14-04-2022 21:01 \\
    \hline\
49.691 & $1.88 \times 10^{-4}$ & --- & --- & 01-01-2022 06:22\\
    &  & 20-02-2022 01:57 & 20-02-2022 00:57 & 19-02-2022 22:57 \\
    &  & 10-04-2022 07:32 & --- & --- \\
\hline\
48.677 & $1.15 \times 10^{-4}$ & --- & --- & 30-12-2021 05:42\\
    &  & 17-02-2022 00:57 & 16-02-2022 23:57 & 16-02-2022 21:57 \\
    &  & 06-04-2022 18:12 & --- & --- \\
\hline\
51.852 & $1.10 \times 10^{-4}$ & 14-11-2021 20:39 & 14-11-2021 19:39 & --- \\
    &  & 05-01-2022 17:06 & --- & --- \\
\hline\
47.703 & $4.45 \times 10^{-5}$ & 10-11-2021 17:05 & &\\
    &  & --- & --- & 27-12-2022 06:57 \\
    &  & 14-02-2022 02:49 & 14-02-2022 01:49 & 13-02-2022 23:49\\
    &  & 02-04-2022 20:42 & 02-04-2022 19:42 & ---
}
\startdata
\enddata
\tablecomments{Transit dates were calculated using the Transit Service Query Form on the NASA Exoplanet Archive. These transit times take into account Daylight Savings time zone changes within the United States.}
\end{deluxetable*}

%\vspace{5mm}
\facilities{TESS, Kepler}


\software{\texttt{exoplanet} \citep{exoplanet2021},
          \texttt{EVEREST 2.0} \citep{luger18},
          \texttt{lightkurve} \citep{lightkurve},
          \texttt{matplotlib} \citep{matplotlib},
          \texttt{PyMC3} \citep{Salvatier16},
          \texttt{starry} \citep{luger19},
          \texttt{theano} \citep{theano},
          \texttt{tica} \citep{fausnaugh20},
          \texttt{batman} \citep{Kreidberg15},
          \texttt{showyourwork!} \citep{luger21},
          \texttt{astropy} \citep{astropy:2013, astropy18},
          \texttt{astroquery}\citep{astroquery19},
          \texttt{numpy} \citep{numpy},
          \texttt{celerite2} \citep{dfm17}
          }



\bibliography{main}
\bibliographystyle{aasjournal}



%% Include this line if you are using the \added, \replaced, \deleted
%% commands to see a summary list of all changes at the end of the article.
%\listofchanges

\end{document}
